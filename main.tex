\documentclass[a4paper]{article}
\usepackage{amsmath,amsfonts,amssymb, commath,fancyhdr,mathtools,bm,dsfont,relsize}
\usepackage[margin= 1.1 in]{geometry}
\usepackage[utf8]{inputenc}
\usepackage{amsthm,graphicx,xifthen,dsfont}
\usepackage[linesnumbered, noend, noline]{algorithm2e} %to write algos
\usepackage{multicol} %package for columns
\usepackage{mdframed} %for framed boxes
\usepackage{pgfplots} % graphs in doc
\usepackage{ragged2e} %for alignment of text
\usepackage{lipsum} %random text
\usepackage{wasysym}
\usepackage{enumerate}
\usepackage{cancel}
\usepackage{float}
\usepackage{marginnote} %package for notes in margins
\usepackage{romannum} %Roman numerals
\usepackage[backref,colorlinks,citecolor=blue,bookmarks=true]{hyperref}

\allowdisplaybreaks

\usepackage[square, numbers, sort]{natbib} %Bibliography stuff
\bibliographystyle{abbrvnat}

\usepackage{mathpazo}
  
%Tikz stuff
\usepackage{calc}
\usepackage{tikz}	%graphics
\usetikzlibrary{patterns}  
\usetikzlibrary{decorations.markings}
\tikzstyle{vertex}=[circle, draw, inner sep=0pt, minimum size=6pt]
\newcommand{\vertex}{\node[vertex]}
\newcounter{Angle}

%Paragraph adn Page formatting commands
\renewcommand{\baselinestretch}{1.15}
\setlength{\parindent}{0em}
\setlength{\parskip}{1em}


%Question Command
\newcommand{\question}[1]
{ \noindent{Problem: \textbf{#1}} \vspace{0.5em}
\hrule \leftskip1em}		


\mdfdefinestyle{answerstyle}{leftmargin=10em,rightmargin=10em}

%Question followed by answer in a box 
\newenvironment{answer}[2]{\begin{description} \item{#1:  #2} \end{description}
\begin{mdframed}[leftmargin = 1em, rightmargin = 1em]}%
{\hspace{5em} \end{mdframed}}


%\begin{ThmOrLemma}{Theorem / Lemma}{Statement of lemma/thm}{
% 	Proof
% }
% \end{ThmOrLemma}
\newenvironment{ThmOrLemma}[2]{
\begin{mdframed}[leftmargin = 1em, rightmargin = 1em]
	\begin{description}[leftmargin = 0em, rightmargin = 0em]
		\begin{flushleft} \item{\textbf{#1}:} #2 \end{flushleft}
 	\end{description}
\begin{flushleft} \textit{Proof}: \ \ }%
{\hspace{5em} \end{flushleft} \qed \end{mdframed}}

\renewcommand{\part}[1] {\vspace{.10in} {#1.}}

%General commands
\newcommand{\id}{\mathds{1}}
\newcommand{\ones}{\vec{\boldsymbol{1}}}
\newcommand{\R}{\mathbb{R}}
\newcommand{\N}{\mathbb{N}}	
\newcommand{\C}{\mathbb{C}}
\newcommand{\Z}{\mathbb{Z}}
\newcommand{\Q}{\mathbb{Q}}
\newcommand{\F}[1]{\ifthenelse{\isempty{#1}}{\mathbb{F}}{\mathbb{F}_{#1}}} %Field_k
\newcommand{\e}{\epsilon}
\newcommand{\nfrac}[2]{\left(\frac{#1}{#2}\right)} %fractions with correct size brackets. 
\newcommand{\brackets}[1]{\left(#1\right)} %Expression with correct size brackets. 
\newcommand{\sbrackets}[1]{\left[#1\right]} %Expression with correct square brackets. 
\newcommand{\seq}[1]{\langle #1 \rangle} %expression for sequences.
\renewcommand{\exp}[1]{\text{exp}\brackets{#1}} %Expression for exponentials 
\newcommand\Eval[3]{\left.#1\right\rvert_{#2}^{#3}} %Evaluation of integral 
\newcommand{\dotp}[2]{\left \langle #1 \ , \ #2 \right \rangle} %A vector.
\DeclareMathOperator*{\argmin}{argmin}
\DeclareMathOperator*{\argmax}{argmax}
\newcommand{\RNum}[1]{\Romannum{#1}}%Roman numeral
\newcommand{\mnote}[1]{\marginnote{\tiny #1}[0cm] }





%Bold face
\newcommand{\bA}{\boldsymbol{A}}
\newcommand{\bB}{\boldsymbol{B}}
\newcommand{\bC}{\boldsymbol{C}}
\newcommand{\bD}{\boldsymbol{D}}
\newcommand{\bE}{\boldsymbol{E}}
\newcommand{\bF}{\boldsymbol{F}}
\newcommand{\bG}{\boldsymbol{G}}
\newcommand{\bH}{\boldsymbol{H}}
\newcommand{\bI}{\boldsymbol{I}}
\newcommand{\bJ}{\boldsymbol{J}}
\newcommand{\bL}{\boldsymbol{L}}
\newcommand{\bP}{\boldsymbol{P}}
\newcommand{\bR}{\boldsymbol{R}}
\newcommand{\bS}{\boldsymbol{S}}
\newcommand{\bT}{\boldsymbol{T}}
\newcommand{\bU}{\boldsymbol{U}}
\newcommand{\bV}{\boldsymbol{V}}
\newcommand{\bX}{\boldsymbol{X}}
\newcommand{\bY}{\boldsymbol{Y}}
\newcommand{\bZ}{\boldsymbol{Z}}
\renewcommand{\bm}{\boldsymbol{m}}
\newcommand{\bw}{\boldsymbol{w}}
\newcommand{\bv}{\boldsymbol{v}}
\newcommand{\bx}{\boldsymbol{x}}
\newcommand{\by}{\boldsymbol{y}}
\newcommand{\br}{\boldsymbol{r}}
\newcommand{\be}{\boldsymbol{e}}

%Curvy for set families
\newcommand{\Ac}{\mathcal{A}}
\newcommand{\Bc}{\mathcal{B}}
\newcommand{\Dc}{\mathcal{D}}
\newcommand{\Xc}{\mathcal{X}}
\newcommand{\Yc}{\mathcal{Y}} 
\newcommand{\Uc}{\mathcal{U}}
\newcommand{\Ic}{\mathcal{I}}
\newcommand{\Cc}{\mathcal{C}}
\newcommand{\Fc}{\mathcal{F}}
\newcommand{\Nc}{\mathcal{N}}
\newcommand{\Lc}{\mathcal{L}}
\newcommand{\Jc}{\mathcal{J}}
\newcommand{\Oc}{\mathcal{O}}
\newcommand{\Tc}{\mathcal{T}}
\newcommand{\Pc}{\mathcal{P}}


%Linear Algebra Commands
\newcommand{\inorm}[1]{\left\Vert #1 \right\Vert}
\newcommand{\twonorm}[1]{\left \Vert #1 \right\Vert_2}
\newcommand{\fnorm}[1]{\Vert #1 \Vert_F}
\newcommand{\iprod}[2]{\ifthenelse{\isempty{#1}}{\langle \cdot,\cdot\rangle}{\langle#1,#2\rangle}}
\newcommand{\trace}[1]{\mathsf{trace}(#1)}
\newcommand{\rank}[1]{\mathsf{rank}(#1)}
\newcommand{\tr}[1]{\mathsf{tr}(#1)}

	
%Probability commands
\newcommand{\Prob}[1]{\textbf{Pr}\left[{#1}\right]}

\newcommand{\Exp}[1]{\textbf{E}\left[{#1}\right]}
\newcommand{\ExSub}[2]{\underset{#1}{\textbf{E}}\left[{#2}\right]}
\newcommand{\PrSub}[2]{\underset{#1}{\textbf{P}}\left[{#2}\right]}
\newcommand{\Ex}{\mathbb{E}}
\newcommand{\Var}[1]{\mathop{\bf Var\/}\left[#1\right]}
\newcommand{\Cov}[2]{\sigma\left(#1,#2\right)}
	
%Analysis Commands
\newcommand{\evals}[2]{\bigg\vert_{#1}^{#2}}
\newcommand{\limit}[2]{\lim_{#1\to#2}}
\newcommand{\dell}[2]{\frac{\partial #1}{\partial #2}} %Partial derivatives[d#1/d#2]
	
%algebra commands
\newcommand{\order}[1]{\vert #1\vert}
\newcommand{\zmod}[1]{\Z/#1\Z}
\newcommand{\gmod}[2]{#1/#2}

%BigO
\newcommand{\bigO}[1]{O\brackets{#1}}
\newcommand{\bigtheta}[1]{\Theta \left(#1 \right)}
\newcommand{\bigomega}[1]{\Omega \left(#1 \right)}

\theoremstyle{definition}
\newtheorem{observation}{Observation}

%theorem environments
\theoremstyle{plain}
\newtheorem*{theorem*}{Theorem}
\newtheorem{theorem}{Theorem}[section]
\newtheorem{lemma}[theorem]{Lemma}
\newtheorem{corollary}[theorem]{Corollary}
\newtheorem{proposition}[theorem]{Proposition}

\theoremstyle{definition}
\newtheorem{definition}[theorem]{Definition}
\newtheorem{remark}[theorem]{Remark}
\newtheorem{fact}[theorem]{Fact}
\newtheorem{algo}[theorem]{Algorithm}

% \setcounter{section}{1}






%document font
\renewcommand*\rmdefault{ppl}
\renewcommand{\baselinestretch}{1.2}

\title{The Log-Rank Conjecture for Linear Threshold Functions composed with AND}
\author{Nicholas Sieger and Aditya Krishnan}
\date{October 8, 2016}

\begin{document}
	\maketitle
	\begin{abstract}
		In this paper, we show that the Log-rank conjecture holds for linear threshold functions (LTFs) of the form $f \circ \wedge$. The result relies on conjecture holding for montone functions of the same form. We show that any LTF can be recursively broken down into montone functions. We further explore the multilinear polynomial of functions of the form $f \circ \wedge$ and give characterizations of its the monomial complexity in terms of the support of the function. 
	\end{abstract}
	\section{Introduction}
	\section{Notation and Definitions}
	We first set our notation for the relevant complexity measures. Consider an arbitrary boolean function $F : \set{0,1}^n\times\set{0,1}^n\to\set{0,1}$ and define $\mathsf{CC}(F)$ denote the communication complexity of $F$. For such an $F$, let $M_F$ denote its communication matrix. 
    
    For each subset $S\subseteq [n]$ define the polynomial $\alpha_S : \set{0,1}^n\to \set{0,1}$ by $\alpha_S(x_1,\dots,x_n) := \prod_{i\in S} x_i$ with $\alpha_\emptyset(x) = 1$. An AND decision tree is then a binary decision tree where each node makes an $\alpha_S$ query for some $S\subseteq [n]$. Such an AND decision tree $T$ decides a boolean function $f : \set{0,1}^n\to\set{0,1}$ if $T$ agrees with $f$ on all inputs $x\in \set{0,1}^n$. The AND decision tree complexity $\mathsf{DT}_{\wedge}(f)$ is the minimum depth of an AND decision tree which decides $f$. 
    
    Given a boolean function $f$, we will frequently write $f$ as a multilinear polynomial. It is shown in \cite{ODonnell2007} that any boolean function $f$ can uniquely written as $f(x) = \sum_{S\subseteq [n]} \hat{\alpha}[S]\alpha_S(x)$ where $\hat{\alpha}[S]$ is the coefficient of $\alpha_S(x)$. We denote by $\mathsf{mon}(f)$ be number of nonzero $\hat{\alpha}[S]$ for a function $f$. The support of a function $f$ is defined as $\mathsf{sup}(f) := f^{-1}(\set{1})$.
    
	We will only consider a subclass of all boolean functions $f: \set{0,1}^n\to \set{0,1}$ which we now define. Let 
    \[
    f(x_1,\dots,x_n) := \begin{cases}
		1 & w_1w_1 + w_2x_2 + \dots + w_nx_n \geq w_0\\
        0 & \text{otherwise}
	\end{cases}
    \] be a linear threshold function (LTF). We will assume without loss of generality that $w_0 \geq 0$. We form our communication function $F(x_1,\dots,x_n,y_1,\dots,y_n) := f(x_1\wedge y_1,x_2\wedge y_2,\dots,x_n\wedge y_n)$. 
    
    Finally, there are several relevant results to state. It is shown in \cite{Buhrman1999} that $\mathsf{rank}(M_F) = \mathsf{mon}(f)$ (Note this result holds for a general $f$, not just LTFs). In addition, it is easy to see that $\mathsf{CC}(F) \leq 2\mathsf{DT}_{\wedge}(f)$. Thus, showing the following claim implies the Log-Rank Conjecture for our class of communication functions $F$.
    \begin{proposition}
    	If $f$ is an LTF, $\mathsf{DF}_{\wedge}(f) \leq \log^c(\mathsf{mon}(f))$ for some $c \in \R^+$.
    \end{proposition}
	\section{Properties of Monomials}
	We characterize the coefficients of the polynomial expansion of a function based on its support. Although seeimingly unmotivated, this characterization will help us in analysing the monomials of LTFs with respect to our proposed query algorithm.

Given a set $S\subseteq [n]$ let us define the set $E^f_S = \set{x\in \sup(f) \ | \ S_x \subseteq S \text{ and $\abs{S_x}$ is even} }$ and $O^f_S = \set{x\in \sup(f) \ | \ S_x \subseteq S \text{ and $\abs{S_x}$ is odd} }$ where $x$ is the characteristic vector of the set $S_v$. Given these notions, we can explicitly calculate the coefficients $\hat{a}[S]$ of the polynomial computing $f$ in terms of $E^f_S$ and $O^f_S$. 

\begin{definition}
	Let us define $\mathds{1}_x$ to be the identity multilinear polynomial that returns $1$ only for input $x$. Then the expansion for $\mathds{1}_x(y)$ is $$\mathds{1}_x(y) = \prod_{i \in S_x} y_i \prod_{j \in \bar{S}_x} (1 - y_j)$$
\end{definition}

We now propose the lemma that characterizes the coefficients. 

\begin{lemma}\label{lemma1}
	For any $S\subseteq [n]$ and function $f: \{0,1\}^n \rightarrow \{0,1\}$, the coefficients $\hat{a}[S]$ are given by 
    \[\hat{a}[S] = -1^{\abs{S}}(\order{E^f_S} - \order{O^f_S})\]
	\begin{proof}
		Every function can be written in terms of the indentity polynomial for each input. That is, $$f(x) = \sum_{y \in \{0, 1\}^n} f(x) \mathds{1}_y(x) = \sum_{y \ | \ f(y) = 1} \mathds{1}_y(x)$$

		We analyse the polynomial expansion for each $\mathds{1}_y(x)$. The expansion is given by the product $\prod_{j \in \bar{S}_y} (1 - x_j)$. The monomials formed by expanding this product correspond to selecting either $-x_j$ or $1$ from each term in the product. Hence we know that every monomial, denoted by a subset $A \subseteq \bar{S}_y$, must occur in the expansion. The coefficient of the monomial corresponding to the set $A \subseteq\bar{S}_y$ is $1$ if $\order{A}$ is even and $-1$ otherwise. Notice that the $\prod_{i \in S_y} x_i$ term exists in every monomial of the expansion of $\mathds{1}_y(x)$. Hence we know a monomial corresponding to some $S \subseteq [n]$ occurs in $\mathds{1}_y(x)$ if and only if $S_y \subseteq S$. Also notice that the coefficient of this monomial is $1$ if $\order{S} - \order{S_y}$ is even and $-1$ otherwise since the parity of the coefficient corresponds to the number of indices picked from $\bar{S}_y$. Hence, summing over all $\mathds{1}_y(x)$ for all $y \in \{0,1\}^n$ gives us the result. 

	\end{proof}
\end{lemma} 
	\section{Breaking Down LTFs}
	In this section we show how an LTF can be broken down into monotone and antimonotone functions. We provide an \textsf{AND} Decision Tree algorithm to compute an LTF based on this decomposition. We use the fact that the Log-Rank holds for monotone functions as and algorithm to compute them as a subroutine for our query algorithm. We then characterize the polynomials of LTFs showing that they're `made up' of the monotone and antimonotone functions we used to compute it.

Throughout this section we will assume that $f$ is a linear threshold function (LTF) of the form $f(x) = w_1x_1 + ... w_nx_n > w_0$. We assume  without loss of generality that $w_0 \geq 0$. For an input $x$, we will denote $W_f(x)$ to denote the set $\{w_i \ | \ x_i = 1\}$ for the LTF $f$. We then define $W^+_f(x)$ to be subset of $W_f(x)$ that has positive weights and correspondingly we define $W^-_f(x)$. Similarly, we define $x^+$ and $x^-$ to be the substrings of $x$ that correspond to the positive and negative weights respectively.

\begin{definition}
	An input $x$ is a $\mathsf{pivot}$ if $f(x) = 1$ and for any $x_i = 1$, flipping $x_i$ to $0$, i.e. $x^{-i}$, causes the function value to flip, i.e. $f(x^{-i}) = 0$. 
\end{definition}

\begin{definition}
	An input $x$ is an $\mathsf{antipivot}$ if $f(x) = 0$ and for any $x_i = 1$ and $w_i < 0$, we have that $f(x^{-i}) = 1$. 
\end{definition}

\begin{fact}\label{fact1}
	There does not exist an input $x'$ and a pivot $x$ such that $S_{x'} \subset S_x$ and $f(x') = 1$. Similarly, this extends to bits corresponding to negative weights in antipivots. 
\end{fact}

The concept of function restrictions easily extends to LTFs. We shall consider the restrictions that restrict some of the input bits to $1$. Let $f_x$ be defined as the function that is restricted to the $0$ bits of $x$. Clearly, the restricted function $f_x$ has weights $\{w_1, ..., w_n\} \backslash W_f(x)$ and has the condition that the linear combination of weights must be more than $w_0 - \sum_{w_i \in W_f(x)} w_i$. Notice that now that this expression is negative for any LTF that has non-zero support. We will define the sets $P_f, A_f$ to be the set of inputs $x$ for $f$ that are pivots and antipivots respectively. 

We are now ready to propose the \textsf{AND} Decision Tree (ADT) algorithm that computes $f(x)$. 

\begin{algorithm}[H]\label{alg1}
\KwIn{$x \in \{0, 1^n\}$}
\While{$f$ is not a constant function}{
	\eIf{$\exists p \in P_f$ such that $p \subset x$}{
		\eIf{$\exists a \in A_{f_p}$ such that $a \subset x$}{$f \leftarrow f_{p + a}$}
		{\KwRet{1}}		
	}{\KwRet{0}}
}
\KwRet{value of $f$}
\caption{LTF Decision Tree Query Algorithm}
\end{algorithm}


We now analyse the communication complexity of the protocol that simulates this \textsf{AND} Decision Tree query algorithm based on the monomial complexity. Let us first show that in lines 2, 3 the pivot and antipivots can be found in polylogarithmic number of queries in the number of monomials of the original function $f$. To do so, we shall define two functions; define the function $f'(x) = 1 \iff \exists p \in P_f \ | \ p \subseteq x$ and for some fixed pivot $p \in P_f$, we define $f^p(x) = 1 \iff \exists a \in A_{f_p} \ | \ a \subseteq x$.  


\begin{proposition}\label{prop1}
	$f'(x)$ and $f^p(x)$ are monotone 
	\begin{proof}
		Since there cannot be a pair of pivots (antipivots) where one is a subset of another from Fact \ref{fact1}, the set $P_f$ $(A_{f_p})$ form a building set for the monotone function. 
	\end{proof}
\end{proposition}

Recall that a building set for a monotone function correspond to the inputs in the support with minimum Hamming weight that are not subsets of each other.

The functions $f', \neg f^p$ can clearly be used directly to compute lines 2, 3 respectively in the algorithm. As per \cite{Buhrman1999,Lovasz1993}, the log-rank conjecture holds for monotone functions and thus there exists communication protocols that compute these functions in polylogarithmic number of communicated bits in the number of their monomials. It is left to show $\mathsf{mon}(f'), \mathsf{mon}( \neg f^p) \leq \mathsf{mon}(f)$.

We shall show a property about monotone functions and their monomials before we reason about the monomial complexity of $f', \neg f^p$.

\begin{lemma}\label{lemma2}
	If $g:\{0, 1\}^n \rightarrow \{0, 1\}$ is a monotone function and let $M \subset \{0,1\}^n$ be the building set for $g$, then we have that $$g(x) = \sum_{k=1}^{|M|} \ \sum_{A \subseteq M, |A| = k} -1^{k-1} \bigcup_{a \in A} a$$ where a string represents a monomial corresponding to its characteristic set and the union of strings is the union of their characteristic sets.

	\begin{proof}
		Consider any input $x$ in $\sup(g)$. Let $A \subseteq M$ such that $\forall a \in A, \ a \subseteq x$. We can assume that $x$ is simply a union of strings from $A$ since the suggested polynomial contains monomials only of that form. It is easy to see that inputs not in the support will be evaluate to $0$ since the function is monotone and the polynomial contains monomials only in the support. Since every monomial $y = \bigcup_{z \in A'}z$ such that $A' \subseteq A$ will evaluate to 1, we will have that the polynomial reduces to $\sum_{i=1}^{|A|} -1^{i-1} \binom{|A|}{i}$. We can see that this sum evaluates to 1 by looking at the binomial expansion of  \\ $ -1((1 - 1)^{|A|} - \binom{|A|}{0}) = (\sum_{k = 0}^{|A|} -1^{k+1} \binom{|A|}{k}) + \binom{|A|}{0}$.
	\end{proof}

\end{lemma}

\begin{proposition}\label{prop2}
	The monomials of the function $f'(x), \neg f^p(x)$ are contained in those of $f(x)$; taking into account the restriction of $f$ in $f^p$. 
	\begin{proof}
		Let us first consider $f'$. Notice that any monomial in $f'$, as per Lemma \ref{lemma2}, is a union of pivots. For all $s \in \sup(f)$ of this form (a union of pivots), $s$ has the property that any $s' \in \sup(f)$ with $s' \subseteq s$ also occurs in $\sup(f')$ and vice versa. Thus, as per Lemma \ref{lemma1}, the quantity $(-1)^{\order{s}}(\order{E_s} - \order{O_s})$ is the same in $f$ and $f'$. 

		Now, let us consider $\neg f^p$ where $p \in P_f$. The support of $\neg f^p$ is characterized by all $s$ such that $\not \exists a \in A_{f_p} \ | \ a \subseteq s$. By the definition of an antipivot, $(p \cup s) \in \sup(f)$. Thus, by Lemma \ref{lemma1}, any monomial $a$, which is a union of antipivots from $A_{f_p}$, must occur as $p \cup a$ in $f$. 
	\end{proof}
\end{proposition}



	\section{Conclusion}
	Using subroutines for monotone functions in an iteration for which the log-rank conjecture holds, we have inductively shown that the log-rank conjecture holds for all LTFs of the form $f \circ \wedge$. Although this result heavily uses a previous result, we conjecture that this algorithm is the optimum  AND Decision Tree algorithm. Our result uses \cite{Buhrman1999} to show the log-rank conjecture for the communication complexity. An interesting problem would be to show it holds for the AND Decision Tree complexity also. 
	\bibliographystyle{alpha}
	\bibliography{paper.bib}
\end{document}	