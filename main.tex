\documentclass[a4paper]{article}
\usepackage{amsmath,amsfonts,amssymb, commath,fancyhdr,mathtools,bm,dsfont,relsize}
\usepackage[margin= 1.1 in]{geometry}
\usepackage[utf8]{inputenc}
\usepackage{amsthm,graphicx,xifthen,dsfont}
\usepackage[linesnumbered, noend, noline]{algorithm2e} %to write algos
\usepackage{multicol} %package for columns
\usepackage{mdframed} %for framed boxes
\usepackage{pgfplots} % graphs in doc
\usepackage{ragged2e} %for alignment of text
\usepackage{lipsum} %random text
\usepackage{wasysym}
\usepackage{enumerate}
\usepackage{cancel}
\usepackage{float}
\usepackage{marginnote} %package for notes in margins
\usepackage{romannum} %Roman numerals


\allowdisplaybreaks

\usepackage[square, numbers, sort]{natbib} %Bibliography stuff
\bibliographystyle{abbrvnat}

\usepackage{mathpazo}
  
%Tikz stuff
\usepackage{calc}
\usepackage{tikz}	%graphics
\usetikzlibrary{patterns}  
\usetikzlibrary{decorations.markings}
\tikzstyle{vertex}=[circle, draw, inner sep=0pt, minimum size=6pt]
\newcommand{\vertex}{\node[vertex]}
\newcounter{Angle}

%Paragraph adn Page formatting commands
\renewcommand{\baselinestretch}{1.15}
\setlength{\parindent}{0em}
\setlength{\parskip}{1em}


%Question Command
\newcommand{\question}[1]
{ \noindent{Problem: \textbf{#1}} \vspace{0.5em}
\hrule \leftskip1em}		


\mdfdefinestyle{answerstyle}{leftmargin=10em,rightmargin=10em}

%Question followed by answer in a box 
\newenvironment{answer}[2]{\begin{description} \item{#1:  #2} \end{description}
\begin{mdframed}[leftmargin = 1em, rightmargin = 1em]}%
{\hspace{5em} \end{mdframed}}


%\begin{ThmOrLemma}{Theorem / Lemma}{Statement of lemma/thm}{
% 	Proof
% }
% \end{ThmOrLemma}
\newenvironment{ThmOrLemma}[2]{
\begin{mdframed}[leftmargin = 1em, rightmargin = 1em]
	\begin{description}[leftmargin = 0em, rightmargin = 0em]
		\begin{flushleft} \item{\textbf{#1}:} #2 \end{flushleft}
 	\end{description}
\begin{flushleft} \textit{Proof}: \ \ }%
{\hspace{5em} \end{flushleft} \qed \end{mdframed}}

\renewcommand{\part}[1] {\vspace{.10in} {#1.}}

%General commands
\newcommand{\id}{\mathds{1}}
\newcommand{\ones}{\vec{\boldsymbol{1}}}
\newcommand{\R}{\mathbb{R}}
\newcommand{\N}{\mathbb{N}}	
\newcommand{\C}{\mathbb{C}}
\newcommand{\Z}{\mathbb{Z}}
\newcommand{\Q}{\mathbb{Q}}
\newcommand{\F}[1]{\ifthenelse{\isempty{#1}}{\mathbb{F}}{\mathbb{F}_{#1}}} %Field_k
\newcommand{\e}{\epsilon}
\newcommand{\nfrac}[2]{\left(\frac{#1}{#2}\right)} %fractions with correct size brackets. 
\newcommand{\brackets}[1]{\left(#1\right)} %Expression with correct size brackets. 
\newcommand{\sbrackets}[1]{\left[#1\right]} %Expression with correct square brackets. 
\newcommand{\seq}[1]{\langle #1 \rangle} %expression for sequences.
\renewcommand{\exp}[1]{\text{exp}\brackets{#1}} %Expression for exponentials 
\newcommand\Eval[3]{\left.#1\right\rvert_{#2}^{#3}} %Evaluation of integral 
\newcommand{\dotp}[2]{\left \langle #1 \ , \ #2 \right \rangle} %A vector.
\DeclareMathOperator*{\argmin}{argmin}
\DeclareMathOperator*{\argmax}{argmax}
\newcommand{\RNum}[1]{\Romannum{#1}}%Roman numeral
\newcommand{\mnote}[1]{\marginnote{\tiny #1}[0cm] }





%Bold face
\newcommand{\bA}{\boldsymbol{A}}
\newcommand{\bB}{\boldsymbol{B}}
\newcommand{\bC}{\boldsymbol{C}}
\newcommand{\bD}{\boldsymbol{D}}
\newcommand{\bE}{\boldsymbol{E}}
\newcommand{\bF}{\boldsymbol{F}}
\newcommand{\bG}{\boldsymbol{G}}
\newcommand{\bH}{\boldsymbol{H}}
\newcommand{\bI}{\boldsymbol{I}}
\newcommand{\bJ}{\boldsymbol{J}}
\newcommand{\bL}{\boldsymbol{L}}
\newcommand{\bP}{\boldsymbol{P}}
\newcommand{\bR}{\boldsymbol{R}}
\newcommand{\bS}{\boldsymbol{S}}
\newcommand{\bT}{\boldsymbol{T}}
\newcommand{\bU}{\boldsymbol{U}}
\newcommand{\bV}{\boldsymbol{V}}
\newcommand{\bX}{\boldsymbol{X}}
\newcommand{\bY}{\boldsymbol{Y}}
\newcommand{\bZ}{\boldsymbol{Z}}
\renewcommand{\bm}{\boldsymbol{m}}
\newcommand{\bw}{\boldsymbol{w}}
\newcommand{\bv}{\boldsymbol{v}}
\newcommand{\bx}{\boldsymbol{x}}
\newcommand{\by}{\boldsymbol{y}}
\newcommand{\br}{\boldsymbol{r}}
\newcommand{\be}{\boldsymbol{e}}

%Curvy for set families
\newcommand{\Ac}{\mathcal{A}}
\newcommand{\Bc}{\mathcal{B}}
\newcommand{\Dc}{\mathcal{D}}
\newcommand{\Xc}{\mathcal{X}}
\newcommand{\Yc}{\mathcal{Y}} 
\newcommand{\Uc}{\mathcal{U}}
\newcommand{\Ic}{\mathcal{I}}
\newcommand{\Cc}{\mathcal{C}}
\newcommand{\Fc}{\mathcal{F}}
\newcommand{\Nc}{\mathcal{N}}
\newcommand{\Lc}{\mathcal{L}}
\newcommand{\Jc}{\mathcal{J}}
\newcommand{\Oc}{\mathcal{O}}
\newcommand{\Tc}{\mathcal{T}}
\newcommand{\Pc}{\mathcal{P}}


%Linear Algebra Commands
\newcommand{\inorm}[1]{\left\Vert #1 \right\Vert}
\newcommand{\twonorm}[1]{\left \Vert #1 \right\Vert_2}
\newcommand{\fnorm}[1]{\Vert #1 \Vert_F}
\newcommand{\iprod}[2]{\ifthenelse{\isempty{#1}}{\langle \cdot,\cdot\rangle}{\langle#1,#2\rangle}}
\newcommand{\trace}[1]{\mathsf{trace}(#1)}
\newcommand{\rank}[1]{\mathsf{rank}(#1)}
\newcommand{\tr}[1]{\mathsf{tr}(#1)}

	
%Probability commands
\newcommand{\Prob}[1]{\textbf{Pr}\left[{#1}\right]}

\newcommand{\Exp}[1]{\textbf{E}\left[{#1}\right]}
\newcommand{\ExSub}[2]{\underset{#1}{\textbf{E}}\left[{#2}\right]}
\newcommand{\PrSub}[2]{\underset{#1}{\textbf{P}}\left[{#2}\right]}
\newcommand{\Ex}{\mathbb{E}}
\newcommand{\Var}[1]{\mathop{\bf Var\/}\left[#1\right]}
\newcommand{\Cov}[2]{\sigma\left(#1,#2\right)}
	
%Analysis Commands
\newcommand{\evals}[2]{\bigg\vert_{#1}^{#2}}
\newcommand{\limit}[2]{\lim_{#1\to#2}}
\newcommand{\dell}[2]{\frac{\partial #1}{\partial #2}} %Partial derivatives[d#1/d#2]
	
%algebra commands
\newcommand{\order}[1]{\vert #1\vert}
\newcommand{\zmod}[1]{\Z/#1\Z}
\newcommand{\gmod}[2]{#1/#2}

%BigO
\newcommand{\bigO}[1]{O\brackets{#1}}
\newcommand{\bigtheta}[1]{\Theta \left(#1 \right)}
\newcommand{\bigomega}[1]{\Omega \left(#1 \right)}

\theoremstyle{definition}
\newtheorem{observation}{Observation}

%theorem environments
\theoremstyle{plain}
\newtheorem*{theorem*}{Theorem}
\newtheorem{theorem}{Theorem}[section]
\newtheorem{lemma}[theorem]{Lemma}
\newtheorem{corollary}[theorem]{Corollary}
\newtheorem{proposition}[theorem]{Proposition}

\theoremstyle{definition}
\newtheorem{definition}[theorem]{Definition}
\newtheorem{remark}[theorem]{Remark}
\newtheorem{fact}[theorem]{Fact}
\newtheorem{algo}[theorem]{Algorithm}

% \setcounter{section}{1}






%document font
\renewcommand*\rmdefault{ppl}
\renewcommand{\baselinestretch}{1.2}

\title{The Log-Rank Conjecture for Linear Threshold Functions composed with $\wedge$}
\author{Nicholas Sieger and Aditya Krishnan}
\date{August 8, 2016}

\begin{document}
	\maketitle
	\begin{abstract}
		In this note, we work towards showing the Log-rank conjecture holds for linear threshold functions (LTFs) of the form $f \circ \wedge$. The result relies on conjecture holding for monotone functions of the same form. We show that any LTF can be recursively broken down into monotone functions. We further explore the multilinear polynomial of functions of the form $f \circ \wedge$ and give characterizations of its the monomial complexity in terms of the support of the function and argue that it is made of monotone parts. Based on the characterization of the polynomial, we suggest an algorithm in the \textsf{AND} decision tree setting to decide a boolean LTF which can be simulated by a communication protocol. Additionally, we show that each round of the communication protocol is at most polynomial in the rank of the function matrix.  
	\end{abstract}
	\section{Introduction}
	Let $F: X\times Y\to \{0,1\}$ be a function (hereafter referred to as a communication function) and $M_F$ be its communication matrix, defined by $(M_F)_{xy} = F(x,y)$. The infamous Log-Rank Conjecture due to Lov\'{a}sz and Saks \cite{Lovasz1988} claims that the deterministic communication complexity of $F$, $\mathsf{CC}(F)$, satisfies $\mathsf{CC}(F) \leq O(\mathsf{polylog}(\rank{M_F}))$. Despite nearly 40 years of work, the best known result for general communication functions is that $\mathsf{CC}(F) \leq O(\sqrt{\rank{M_F}\log(\rank{M_F})})$. Rather than tackle arbitrary communication functions, a number of authors focus on restricted classes of functions. One such class is functions of the form $f(g(x_1,y_1),g(x_2,y_2),\dots,f(x_n,y_n))$ where $f : \{0,1\}^n \to \{0,1\}$ and $g : \{0,1\}\times\{0,1\}\to\{0,1\}$ is a ``gadget'' function, which we denote by $f \circ g^{\oplus n}$. By fixing $g = \otimes$, \cite{Tsang2013} and \cite{Lovett2016} began to verify the Log-Rank Conjecture in this restricted setting, and gave a close connection to XOR decision tree complexity. The connection between communication and decision tree complexity was also developed by \cite{Goos2015} in a powerful query-to-communication lifting theorem. Furthermore, \cite{Goos2015} used their lifting theorem to show an $\Omega(\log^2)$ lower bound to $\mathsf{CC}(F)$.

\subsection{Our Work}

We follow the direction proposed in \cite{Lovett2016} and consider functions of the form $f\circ \wedge^{\otimes n}$. In particular, we fix $f$ to be an arbitrary \textit{Linear Threshold Function} (LTF). Given our nicely restricted setting, we give a characterization of the multilinear polynomials which express an LTF, and propose an algorithm to build an AND decision tree from a communication protocol for $f\circ \wedge^{\otimes n}$ in the hopes of proving the Log-Rank Conjecture in this new setting. We compute the function by breaking the LTF into monotone and antimonotone counterparts. Then, by analysing the polynomial for the function, we show that the monomials of the polynomial consist of exactly the monotone and antimonotone functions that we compute in our proposed algorithm. Any \textsf{AND} decision tree algorithm for the $\wedge$ gadget can be simulated by a communication protocol with atmost $2\mathsf{DT}_\wedge(f)$ bits communicated by Alice sending her required bit of the input followed by Bob sending the result of the querying the appropriate bit. 
\begin{theorem}\label{mainresult}
	Let $f : \{0,1\}^n\to\{0,1\}$ be a linear threshold function. Form a communication function $F: \{0,1\}^n\times \{0,1\}^n\to\{0,1\}$ by $F(x_1,\dots,x_n,y_1,\dots,y_n)  = f(x_1 \wedge y_1,x_2\wedge y_2,\dots,x_n\wedge y_n)$. Let  $M_F$ be the communication matrix of $F$. Then $\mathsf{CC}(F) \leq \mathsf{polylog}(\rank{M_F})$. 
\end{theorem}

	\section{Notation and Definitions}
	We first set our notation for the relevant complexity measures. Consider an arbitrary boolean function $F : \set{0,1}^n\times\set{0,1}^n\to\set{0,1}$ and define $\mathsf{CC}(F)$ denote the \textit{communication complexity} of $F$. For such an $F$, let $M_F$ denote its communication matrix. 
    
    For each subset $S\subseteq [n]$ define the polynomial $\alpha_S : \set{0,1}^n\to \set{0,1}$ by $\alpha_S(x_1,\dots,x_n) := \prod_{i\in S} x_i$ with $\alpha_\emptyset(x) = 1$. An \textit{AND decision tree} is then a binary decision tree where each node makes an $\alpha_S$ query for some $S\subseteq [n]$. Such an AND decision tree $T$ decides a boolean function $f : \set{0,1}^n\to\set{0,1}$ if $T$ agrees with $f$ on all inputs $x\in \set{0,1}^n$. The \textit{AND decision tree complexity} $\mathsf{DT}_{\wedge}(f)$ is the minimum depth of an AND decision tree which decides $f$. 
    
    Given a boolean function $f$, we will frequently write $f$ as a multilinear polynomial. It is shown in \cite{ODonnell2007} that any boolean function $f$ can uniquely written as $f(x) = \sum_{S\subseteq [n]} \hat{\alpha}[S]\alpha_S(x)$ where $\hat{\alpha}[S]$ is the coefficient of $\alpha_S(x)$. We denote by $\mathsf{mon}(f)$ be number of nonzero $\hat{\alpha}[S]$ for a function $f$. The \textit{support} of a function $f$ is defined as $\mathsf{sup}(f) := f^{-1}(\set{1})$.
    
	We will only consider a subclass of all boolean functions $f: \set{0,1}^n\to \set{0,1}$ which we now define. Let 
    \[
    f(x_1,\dots,x_n) := \begin{cases}
		1 & w_1w_1 + w_2x_2 + \dots + w_nx_n \geq w_0\\
        0 & \text{otherwise}
	\end{cases}
    \] be a \textit{linear threshold function} (LTF). We will assume without loss of generality that $w_0 \geq 0$. We form our communication function $F(x_1,\dots,x_n,y_1,\dots,y_n) := f(x_1\wedge y_1,x_2\wedge y_2,\dots,x_n\wedge y_n)$. 
    
    Finally, there are several relevant results to state. It is shown in \cite{Buhrman1999} that $\mathsf{rank}(M_F) = \mathsf{mon}(f)$ (Note this result holds for a general $f$, not just LTFs). In addition, it is easy to see that $\mathsf{CC}(F) \leq 2\mathsf{DT}_{\wedge}(f)$. Thus, showing the following claim implies the Log-Rank Conjecture for our class of communication functions $F$.
    \begin{proposition}
    	If $f$ is an LTF, $\mathsf{DF}_{\wedge}(f) \leq \log^c(\mathsf{mon}(f))$ for some $c \in \R^+$.
    \end{proposition}
	\section{Properties of Monomials}
	We characterize the coefficients of the polynomial expansion of a function based on its support. Although seeimingly unmotivated, this characterization will help us in analysing the monomials of LTFs with respect to our proposed query algorithm.

Given a set $S\subseteq [n]$ let us define the set $E^f_S = \set{x\in \sup(f) \ | \ S_x \subseteq S \text{ and $\abs{S_x}$ is even} }$ and $O^f_S = \set{x\in \sup(f) \ | \ S_x \subseteq S \text{ and $\abs{S_x}$ is odd} }$ where $x$ is the characteristic vector of the set $S_v$. Given these notions, we can explicitly calculate the coefficients $\hat{a}[S]$ of the polynomial computing $f$ in terms of $E^f_S$ and $O^f_S$. 

\begin{definition}
	Let us define $\mathds{1}_x$ to be the identity multilinear polynomial that returns $1$ only for input $x$. Then the expansion for $\mathds{1}_x(y)$ is $$\mathds{1}_x(y) = \prod_{i \in S_x} y_i \prod_{j \in \bar{S}_x} (1 - y_j)$$
\end{definition}

We now propose the lemma that characterizes the coefficients. 

\begin{lemma}\label{lemma1}
	For any $S\subseteq [n]$ and function $f: \{0,1\}^n \rightarrow \{0,1\}$, the coefficients $\hat{a}[S]$ are given by 
    \[\hat{a}[S] = -1^{\abs{S}}(\order{E^f_S} - \order{O^f_S})\]
	\begin{proof}
		Every function can be written in terms of the indentity polynomial for each input. That is, $$f(y) = \sum_{x \in \{0, 1\}^n} f(y) \mathds{1}_x(y) = \sum_{x \ | \ f(x) = 1} \mathds{1}_x(y)$$

		We analyse the polynomial expansion for each $\mathds{1}_y(x)$. The expansion is given by the product $\prod_{j \in \bar{S}_y} (1 - x_j)$. The monomials formed by expanding this product correspond to selecting either $-x_j$ or $1$ from each term in the product. Hence we know that every monomial, denoted by a subset $A \subseteq \bar{S}_y$, must occur in the expansion. The coefficient of the monomial corresponding to the set $A \subseteq\bar{S}_y$ is $1$ if $\order{A}$ is even and $-1$ otherwise. Notice that the $\prod_{i \in S_y} x_i$ term exists in every monomial of the expansion of $\mathds{1}_y(x)$. Hence we know a monomial corresponding to some $S \subseteq [n]$ occurs in $\mathds{1}_y(x)$ only if $S_y \subseteq S$. Also notice that the coefficient of the monomial corresponding to $S \subseteq [n]$ in $\mathds{1}_y(x)$ is $1$ if $\order{S} - \order{S_y}$ is even and $-1$ otherwise. Hence, summing over all $\mathds{1}_y(x)$ for all $y \in \{0,1\}^n$ gives us the result. 

	\end{proof}
\end{lemma} 
	\section{Breaking Down LTFs}
	In this section we show how an LTF can be broken down into monotone and antimonotone functions. We provide an AND Decision Tree algorithm to compute an LTF based on this decomposition and then conclude that there exists a communication protocol that computes the LTF, formed by bitwise $\wedge$ of the inputs to Alice and Bob, that is polylogarithmic in the monomial complexity of the LTF.  Since $\text{rank}({M_{f_{\wedge}}}) = mon(f_{\wedge})$ from \cite{Buhrman1999}, we have that the Log-rank holds for LTFs of the mentioned form.

Throughout this section we will assume that $f$ is a linear threshold function (LTF) of the form $f(x) = w_1x_1 + ... w_nx_n > w_0$. We assume  without loss of generality that $w_0 \geq 0$. 

For an input $x$, we will denote $W_f(x)$ to denote the set $\{w_i \ | \ x_i = 1\}$ for the LTF $f$. Naturally, we define $W^+_f(x)$ to be subset of $W_f(x)$ that has positive weights and correspondingly we define $W^-_f(x)$. Similarly, we define $x^+$ and $x^-$ to be the substrings that correspond to the weights.

\begin{definition}
	An input $x$ is a $pivot$ if $f(x) = 1$ and for any $x_i = 1$, flipping $x_i$ to $0$, $x^{-i}$, causes the function value to flip, $f(x^{-i}) = 0$. 
\end{definition}

\begin{definition}
	An input $x$ is an $antipivot$ if $f(x) = 0$ and for any $x_i = 1, w_i < 0$, we have that $f(x^{-i}) = 1$. 
\end{definition}

\begin{fact}\label{fact1}
	There does not exist an input $x'$ and a pivot $x$ such that $S_{x'} \subset S_x$ and $f(x') = 1$. Similarly, this extends to bits corresponding to negative weights in antipivots. 
\end{fact}

The concept of function restrictions easily extends to LTFs. We shall consider the restrictions that restrict some of the input bits to $1$. Let $f_x$ be defined as the function that is restricted to the $0$ bits of $x$. Clearly, the restricted function $f_x$ has weights $\{w_1, ..., w_n\} \backslash W_f(x)$ and has the condition that the linear combination of weights must be more than $w_0 - \sum_{w_i \in W_f(x)} w_i$. Notice that now that this expression is negative. We will define $P_f, A_f$ to the set of inputs $x$ for $f$ that are pivots and antipivots respectively. 

We are now ready to propose the AND Decision Tree (ADT) algorithm that computes $f(x)$. 

\begin{algorithm}[H]\label{alg1}
\KwIn{$x \in \{0, 1^n\}$}
\While{$f$ is not a constant function}{
	\eIf{$\exists p \in P_f$ such that $p \subset x$}{
		\eIf{$\exists a \in A_{f_p}$}{$f \leftarrow f_{p + a}$}
		{\KwRet{1}}		
	}{\KwRet{0}}
}
\KwRet{value of $f$}
\caption{LTF Decision Tree Query Algorithm}
\end{algorithm}


We now analyse the communication complexity of the protocol that simulates this AND Decision Tree query algorithm based on the monomial complexity. By inducting on $n$, the number of input variables, we shall inductively assume that $CC(f(x \wedge y)) \leq \log^c(mon(f))$. It is then left to show that in lines 2, 3 the pivot and antipivots can be found in polylogarithmic number of queries in the number of monomials. We shall define 2 such functions. Define the function $f'(x) = 1 \iff \exists p \in P_f \ | \ p \subseteq x$ and for some fixed pivot $p \in P_f$, we define $f^p(x) = 1 \iff \exists a \in A_{f_p} \ | \ a \subseteq x$.  


\begin{proposition}\label{prop1}
	$f'(x)$ and $f^p(x)$ are montone 
	\begin{proof}
		Since there cannot be a pair of pivots (antipivots) where one is a subset of another, Fact \ref{fact1}, the set $P_f$ $(A_{f_p})$ forms a building set for the montone function. 
	\end{proof}
\end{proposition}

The functions $f', \neg f^p$ can clearly be used directly to compute lines 2, 3 respectively in the algorithm. As per \cite{Buhrman1999,Lovasz1993}, the log-rank conjecture holds for monotone functions and thus there exists communication protocols that compute these functions in polylogarithmic number of communicated bits in the number of their monomials. It is left to show $mon(f'), mon( \neg f^p) \leq mon(f)$.

We shall show a property about montone functions and their monomials before we reason about the monomial complexity of $f', \neg f^p$.

\begin{lemma}\label{lemma2}
	If $g:\{0, 1\}^n \rightarrow \{0, 1\}$ is a monotone function and let $M \subset \{0,1\}^n$ be the building set for $g$, then we have that $$g(x) = \sum_{k=1}^{|M|} \ \sum_{A \subseteq M, |A| = k} -1^{k-1} \bigcup_{x \in A} x$$ where a string represents a monomial and the union of strings is the union of their characteristic sets.

	\begin{proof}
		Consider any input $x$ in $\sup(g)$. Let $A \subseteq M$ such that $\forall a \in A, \ a \subseteq x$. We can assume that $x$ is simply a union of strings from $A$ since the suggested polynomial contains monomials only of that form. Since any monomial of the form $y \not \subseteq x$ will evaluate to 0 and every monomial $y = \bigcup_{z \in A'}z, \ A' \subseteq A$ will evaluate to 1, we will have that the polynomial reduces to $\sum_{i=1}^{|A|} -1^{i-1} \binom{|A|}{i}$. We can see that this sum evaluates to 1 by looking at the binomial expansion of  \\ $ -1((1 - 1)^{|A|} - \binom{|A|}{0}) = (\sum_{k = 0}^{|A|} -1^{k+1} \binom{|A|}{k}) + \binom{|A|}{0}$.
	\end{proof}

\end{lemma}

\begin{proposition}\label{prop2}
	The monomials of the function $f'(x), \neg f^p(x)$ are contained in those of $f(x)$; taking into account the restriction of $f$ in $f^p$. 
	\begin{proof}
		Let us first consider $f'$. Notice that any monomial in $f'$, as per Lemma \ref{lemma2}, is a union of pivots. For all $s \in \sup(f)$ of this form (a union of pivots), $s$ has the property that any $s' \in \sup(f), \ s' \subseteq s$ also occurs in $\sup(f')$ and vice versa. Thus, as per \ref{lemma1}, the quantity $(-1)^{\order{s}}(\order{E_s} - \order{O_s})$ is the same in $f$ and $f'$. 

		Now, let us consider $\neg f^p$ where $p \in P_f$. The support of $\neg f^p$ is characterized by all $s$ such that $\not \exists a \in A_{f_p} \ | \ a \subseteq s$. By the definition of an antipivot, $(p \cup s) \in \sup(f)$. Thus, by Lemma \ref{lemma1}, any monomial $a$, which is a union of antipivots from $A_{f_p}$, must occur as $p \cup a$ in $f$. 
	\end{proof}
\end{proposition}


Combining Proposition \ref{prop2} with Algorithm \ref{alg1} we have inductively shown that $CC(f \circ \wedge) \leq \log^c(mon(f))$ for $c \in O(1)$. 
	\section{Conclusion}
	It is left to show in the analysis of our algorithm that the monomial complexity of our function drops sufficiently fast enough. Although unable to complete this analysis we conjecture that this query algorithm when simulated by a communication protocol is $\textsf{poly}\log(\textsf{rank } M_f)$. Notice that a subroutine used by our algorithm is a communication protocol that computes a monotone function. Hence, it is not clear that this subroutine can be computed with few queries in the $\textsf{DT}_\wedge$ setting.Thus, an interesting problem would be to show that the Log-Rank conjecture holds for the AND Decision Tree complexity also. A stepping-stone in showing this would be to give an explicit algorithm for monotone functions for which $\textsf{DT}_\wedge(f) \leq \textsf{poly}\log(\textsf{mon } f)$. 

	\newpage
	
	\bibliographystyle{alpha}
	\bibliography{paper}
\end{document}	