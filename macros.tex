\usepackage{amsmath,amsfonts,amssymb, commath,fancyhdr,mathtools,bm,dsfont,relsize}
\usepackage[margin= 1.1 in]{geometry}
\usepackage[utf8]{inputenc}
\usepackage{amsthm,graphicx,xifthen,dsfont}
\usepackage[linesnumbered, noend, noline]{algorithm2e} %to write algos
\usepackage{multicol} %package for columns
\usepackage{mdframed} %for framed boxes
\usepackage{pgfplots} % graphs in doc
\usepackage{ragged2e} %for alignment of text
\usepackage{lipsum} %random text
\usepackage{wasysym}
\usepackage{enumerate}
\usepackage{cancel}
\usepackage{float}
\usepackage{marginnote} %package for notes in margins
\usepackage{romannum} %Roman numerals


\allowdisplaybreaks

\usepackage[square, numbers, sort]{natbib} %Bibliography stuff
\bibliographystyle{abbrvnat}

\usepackage{mathpazo}
  
%Tikz stuff
\usepackage{calc}
\usepackage{tikz}	%graphics
\usetikzlibrary{patterns}  
\usetikzlibrary{decorations.markings}
\tikzstyle{vertex}=[circle, draw, inner sep=0pt, minimum size=6pt]
\newcommand{\vertex}{\node[vertex]}
\newcounter{Angle}

%Paragraph adn Page formatting commands
\renewcommand{\baselinestretch}{1.15}
\setlength{\parindent}{0em}
\setlength{\parskip}{1em}


%Question Command
\newcommand{\question}[1]
{ \noindent{Problem: \textbf{#1}} \vspace{0.5em}
\hrule \leftskip1em}		


\mdfdefinestyle{answerstyle}{leftmargin=10em,rightmargin=10em}

%Question followed by answer in a box 
\newenvironment{answer}[2]{\begin{description} \item{#1:  #2} \end{description}
\begin{mdframed}[leftmargin = 1em, rightmargin = 1em]}%
{\hspace{5em} \end{mdframed}}


%\begin{ThmOrLemma}{Theorem / Lemma}{Statement of lemma/thm}{
% 	Proof
% }
% \end{ThmOrLemma}
\newenvironment{ThmOrLemma}[2]{
\begin{mdframed}[leftmargin = 1em, rightmargin = 1em]
	\begin{description}[leftmargin = 0em, rightmargin = 0em]
		\begin{flushleft} \item{\textbf{#1}:} #2 \end{flushleft}
 	\end{description}
\begin{flushleft} \textit{Proof}: \ \ }%
{\hspace{5em} \end{flushleft} \qed \end{mdframed}}

\renewcommand{\part}[1] {\vspace{.10in} {#1.}}

%General commands
\newcommand{\id}{\mathds{1}}
\newcommand{\ones}{\vec{\boldsymbol{1}}}
\newcommand{\R}{\mathbb{R}}
\newcommand{\N}{\mathbb{N}}	
\newcommand{\C}{\mathbb{C}}
\newcommand{\Z}{\mathbb{Z}}
\newcommand{\Q}{\mathbb{Q}}
\newcommand{\F}[1]{\ifthenelse{\isempty{#1}}{\mathbb{F}}{\mathbb{F}_{#1}}} %Field_k
\newcommand{\e}{\epsilon}
\newcommand{\nfrac}[2]{\left(\frac{#1}{#2}\right)} %fractions with correct size brackets. 
\newcommand{\brackets}[1]{\left(#1\right)} %Expression with correct size brackets. 
\newcommand{\sbrackets}[1]{\left[#1\right]} %Expression with correct square brackets. 
\newcommand{\seq}[1]{\langle #1 \rangle} %expression for sequences.
\renewcommand{\exp}[1]{\text{exp}\brackets{#1}} %Expression for exponentials 
\newcommand\Eval[3]{\left.#1\right\rvert_{#2}^{#3}} %Evaluation of integral 
\newcommand{\dotp}[2]{\left \langle #1 \ , \ #2 \right \rangle} %A vector.
\DeclareMathOperator*{\argmin}{argmin}
\DeclareMathOperator*{\argmax}{argmax}
\newcommand{\RNum}[1]{\Romannum{#1}}%Roman numeral
\newcommand{\mnote}[1]{\marginnote{\tiny #1}[0cm] }





%Bold face
\newcommand{\bA}{\boldsymbol{A}}
\newcommand{\bB}{\boldsymbol{B}}
\newcommand{\bC}{\boldsymbol{C}}
\newcommand{\bD}{\boldsymbol{D}}
\newcommand{\bE}{\boldsymbol{E}}
\newcommand{\bF}{\boldsymbol{F}}
\newcommand{\bG}{\boldsymbol{G}}
\newcommand{\bH}{\boldsymbol{H}}
\newcommand{\bI}{\boldsymbol{I}}
\newcommand{\bJ}{\boldsymbol{J}}
\newcommand{\bL}{\boldsymbol{L}}
\newcommand{\bP}{\boldsymbol{P}}
\newcommand{\bR}{\boldsymbol{R}}
\newcommand{\bS}{\boldsymbol{S}}
\newcommand{\bT}{\boldsymbol{T}}
\newcommand{\bU}{\boldsymbol{U}}
\newcommand{\bV}{\boldsymbol{V}}
\newcommand{\bX}{\boldsymbol{X}}
\newcommand{\bY}{\boldsymbol{Y}}
\newcommand{\bZ}{\boldsymbol{Z}}
\renewcommand{\bm}{\boldsymbol{m}}
\newcommand{\bw}{\boldsymbol{w}}
\newcommand{\bv}{\boldsymbol{v}}
\newcommand{\bx}{\boldsymbol{x}}
\newcommand{\by}{\boldsymbol{y}}
\newcommand{\br}{\boldsymbol{r}}
\newcommand{\be}{\boldsymbol{e}}

%Curvy for set families
\newcommand{\Ac}{\mathcal{A}}
\newcommand{\Bc}{\mathcal{B}}
\newcommand{\Dc}{\mathcal{D}}
\newcommand{\Xc}{\mathcal{X}}
\newcommand{\Yc}{\mathcal{Y}} 
\newcommand{\Uc}{\mathcal{U}}
\newcommand{\Ic}{\mathcal{I}}
\newcommand{\Cc}{\mathcal{C}}
\newcommand{\Fc}{\mathcal{F}}
\newcommand{\Nc}{\mathcal{N}}
\newcommand{\Lc}{\mathcal{L}}
\newcommand{\Jc}{\mathcal{J}}
\newcommand{\Oc}{\mathcal{O}}
\newcommand{\Tc}{\mathcal{T}}
\newcommand{\Pc}{\mathcal{P}}


%Linear Algebra Commands
\newcommand{\inorm}[1]{\left\Vert #1 \right\Vert}
\newcommand{\twonorm}[1]{\left \Vert #1 \right\Vert_2}
\newcommand{\fnorm}[1]{\Vert #1 \Vert_F}
\newcommand{\iprod}[2]{\ifthenelse{\isempty{#1}}{\langle \cdot,\cdot\rangle}{\langle#1,#2\rangle}}
\newcommand{\trace}[1]{\mathsf{trace}(#1)}
\newcommand{\rank}[1]{\mathsf{rank}(#1)}
\newcommand{\tr}[1]{\mathsf{tr}(#1)}

	
%Probability commands
\newcommand{\Prob}[1]{\textbf{Pr}\left[{#1}\right]}

\newcommand{\Exp}[1]{\textbf{E}\left[{#1}\right]}
\newcommand{\ExSub}[2]{\underset{#1}{\textbf{E}}\left[{#2}\right]}
\newcommand{\PrSub}[2]{\underset{#1}{\textbf{P}}\left[{#2}\right]}
\newcommand{\Ex}{\mathbb{E}}
\newcommand{\Var}[1]{\mathop{\bf Var\/}\left[#1\right]}
\newcommand{\Cov}[2]{\sigma\left(#1,#2\right)}
	
%Analysis Commands
\newcommand{\evals}[2]{\bigg\vert_{#1}^{#2}}
\newcommand{\limit}[2]{\lim_{#1\to#2}}
\newcommand{\dell}[2]{\frac{\partial #1}{\partial #2}} %Partial derivatives[d#1/d#2]
	
%algebra commands
\newcommand{\order}[1]{\vert #1\vert}
\newcommand{\zmod}[1]{\Z/#1\Z}
\newcommand{\gmod}[2]{#1/#2}

%BigO
\newcommand{\bigO}[1]{O\brackets{#1}}
\newcommand{\bigtheta}[1]{\Theta \left(#1 \right)}
\newcommand{\bigomega}[1]{\Omega \left(#1 \right)}

\theoremstyle{definition}
\newtheorem{observation}{Observation}

%theorem environments
\theoremstyle{plain}
\newtheorem*{theorem*}{Theorem}
\newtheorem{theorem}{Theorem}[section]
\newtheorem{lemma}[theorem]{Lemma}
\newtheorem{corollary}[theorem]{Corollary}
\newtheorem{proposition}[theorem]{Proposition}

\theoremstyle{definition}
\newtheorem{definition}[theorem]{Definition}
\newtheorem{remark}[theorem]{Remark}
\newtheorem{fact}[theorem]{Fact}
\newtheorem{algo}[theorem]{Algorithm}

% \setcounter{section}{1}




