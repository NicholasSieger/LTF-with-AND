Let $F: X\times Y\to \{0,1\}$ be a function (hereafter referred to as a communication function) and $M_F$ be its communication matrix, defined by $(M_F)_{xy} = F(x,y)$. The infamous Log-Rank Conjecture due to Lov\'{a}sz and Saks \cite{Lovasz1988} claims that the deterministic communication complexity of $F$, $\mathsf{CC}(F)$, satisfies $\mathsf{CC}(F) \leq O(\mathsf{polylog}(\rank{M_F}))$. Despite nearly 40 years of work, the best known result for general communication functions is that $\mathsf{CC}(F) \leq O(\sqrt{\rank{M_F}\log(\rank{M_F})})$. Rather than tackle arbitrary communication functions, a number of authors focus on restricted classes of functions. One such class is functions of the form $f(g(x_1,y_1),g(x_2,y_2),\dots,f(x_n,y_n))$ where $f : \{0,1\}^n \to \{0,1\}$ and $g : \{0,1\}\times\{0,1\}\to\{0,1\}$ is a ``gadget'' function, which we denote by $f \circ g^{\oplus n}$. By fixing $g = \otimes$, \cite{Tsang2013} and \cite{Lovett2016} began to verify the Log-Rank Conjecture in this restricted setting, and gave a close connection to XOR decision tree complexity. The connection between communication and decision tree complexity was also developed by \cite{Goos2015} in a powerful query-to-communication lifting theorem. Furthermore, \cite{Goos2015} used their lifting theorem to show an $\Omega(\log^2)$ lower bound to $\mathsf{CC}(F)$.

\subsection{Our Work}

We follow the direction proposed in \cite{Lovett2016} and consider functions of the form $f\circ \wedge^{\otimes n}$. In particular, we fix $f$ to be an arbitrary \textit{Linear Threshold Function} (LTF). Given our nicely restricted setting, we give a characterization of the multilinear polynomials which express an LTF, and propose an algorithm to build an AND decision tree from a communication protocol for $f\circ \wedge^{\otimes n}$ in the hopes of proving the Log-Rank Conjecture in this new setting. We compute the function by breaking the LTF into monotone and antimonotone counterparts. Then, by analysing the polynomial for the function, we show that the monomials of the polynomial consist of exactly the monotone and antimonotone functions that we compute in our proposed algorithm. Any \textsf{AND} decision tree algorithm for the $\wedge$ gadget can be simulated by a communication protocol with atmost $2\mathsf{DT}_\wedge(f)$ bits communicated by Alice sending her required bit of the input followed by Bob sending the result of the querying the appropriate bit. 
\begin{theorem}\label{mainresult}
	Let $f : \{0,1\}^n\to\{0,1\}$ be a linear threshold function. Form a communication function $F: \{0,1\}^n\times \{0,1\}^n\to\{0,1\}$ by $F(x_1,\dots,x_n,y_1,\dots,y_n)  = f(x_1 \wedge y_1,x_2\wedge y_2,\dots,x_n\wedge y_n)$. Let  $M_F$ be the communication matrix of $F$. Then $\mathsf{CC}(F) \leq \mathsf{polylog}(\rank{M_F})$. 
\end{theorem}
