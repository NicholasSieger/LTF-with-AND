Let $F: X\times Y\to \{0,1\}$ be a function (hereafter referred to as a communication function) and $M_F$ be its communication matrix, defined by $(M_F)_{xy} = F(x,y)$. The infamous Log-Rank Conjecture due to Lova\'{a}sz and Saks \cite{Lovasz1988} claims that the deterministic communication complexity of $F$, $CC(F)$, satisfies $CC(F) \leq O(\mathsf{polylog}(\rank{M_F}))$. Despite nearly 40 years of work, the best known result for general communication functions is that $CC(F) \leq O(\sqrt{\rank{M_F}\log(\rank{M_F})})$. Rather than tackle arbitrary commmunication functions, a number of authors focus on restricted classes of functions. One such class is functions of the form $f(g(x_1,y_1),g(x_2,y_2),\dots,f(x_n,y_n))$ where $f : \{0,1\}^n \to \{0,1\}$ and $g : \{0,1\}\times\{0,1\}\to\{0,1\}$ is a ``gadget'' function, which we denote by $f \circ g^{\oplus n}$. By fixing $g = \otimes$, \cite{Tsang2013} and \cite{Lovett2016} began to verify the Log-Rank Conjecutre in this restricted setting, and gave a close connection to XOR decision tree complexity. The connection between communicaiton and decision tree complexity was also developed by \cite{Goos2015} in a powerful query-to-communication lifting theorem. Furthermore, \cite{Goos2015} used their lifting theorem to show that the $\mathsf{polylog}$ in the Log-Rank Conjecture is $\Omega(\log^2)$.

We follow the direction proposed in \cite{Lovett2016} and consider functions of the form $f\circ \wedge^{\otimes n}$. In particular, we fix $f$ to be an arbitrary \textit{Linear Threshold Function}(LTF). Given our nicely restricted setting, we give a characterization of the multilinear polynomials which express an LTF, and propose an algorithm to build an AND decision tree from a communication protocol for $f\circ \wedge^{\otimes n}$ in the hopes of proving the Log-Rank Conjecture in this new setting. 
\begin{theorem}\label{mainresult}
	Let $f : \{0,1\}^n\to\{0,1\}$ be a linear threshold function. Form a communication function $F: \{0,1\}^n\times \{0,1\}^n\to\{0,1\}$ by $F(x_1,\dots,x_n,y_1,\dots,y_n)  = f(x_1 \wedge y_1,x_2\wedge y_2,\dots,x_n\wedge y_n)$. Let  $M_F$ be the communication matrix of $F$. Then $CC(F) \leq \mathsf{polylog}(\rank{M_F})$. 
\end{theorem}

Our note is organizaed as follows. First, we introduce our necessary definitions and notation. Then, we develop our characterization of the polynomials which express an LTF. Finally, we propose our simulation algorithm and begin its analysis. 